
\documentclass[journal]{IEEEtran}

%All packages used goes here:
\usepackage{lipsum}
\usepackage{graphicx}



\begin{document}

\title{Your Title\\ New Line}


\author{Student Name - \textit{Student number}\\
        Student Name - \textit{Student number}\\
        Student Name - \textit{Student number}\\
}

\markboth{EEE1006F Semester Project - Electrical Engineering Department - UCT - \today}%
{Shell \MakeLowercase{\textit{et al.}}: Bare Demo of IEEEtran.cls for IEEE Journals}% <- leave this as is

\maketitle

\begin{abstract}
The abstract goes here.
\end{abstract}

\IEEEpeerreviewmaketitle



\section{Introduction}

\IEEEPARstart{T}{his} 
\lipsum[10] % <- remove this nonsense text



\subsection{Subsection Heading Here}
Subsection text here. 

%%%%%%%%%%%%%%%%%%%%%%%%%%%%%%
%Figures: See: https://www.overleaf.com/learn/latex/Inserting_Images
%%%%%%%%%%%%%%%%%%%%%%%%%%%%%%

%Here is how you would add an image. First upload the image on the left of your screen by pressing on the up arrow next to the folder icon.
\begin{figure}[h!]
\centering % <- centre the image
\includegraphics[width=8cm]{example.jpg} % <- set the size and file path/name
\caption{Example of an image}
\label{fig:circ} %<- give the figure a name to refer to later
\end{figure}

%%%%%%%%%%%%%%%%%%%%%%%%%%%%%%
%References:
%%%%%%%%%%%%%%%%%%%%%%%%%%%%%%

Refer to Figure \ref{fig:circ} like this. Refer to Equation \ref{Eq:E} like this. Refer to Table \ref{tab:cells} like this.

Cite a reference like this: blah blah blah \cite{examplecite} %  <- see the name at the end under bibliography

%%%%%%%%%%%%%%%%%%%%%%%%%%%%%%
%Paragraphs and pages:
%%%%%%%%%%%%%%%%%%%%%%%%%%%%%%
\newpage %<- new page

Blah\\\\ %  <- use this at the end of a paragraph to create a paragraph space

Blah\\ %  <- use this at the end of a paragraph to create a new paragraph without space

%%%%%%%%%%%%%%%%%%%%%%%%%%%%%%
%Maths: See: https://www.codecogs.com/latex/eqneditor.php
%%%%%%%%%%%%%%%%%%%%%%%%%%%%%%
In line maths is done like this $P=I^{2}R$, otherwise use:
\begin{equation}
E=mc^{2}
\label{Eq:E} %<- give the figure a name to refer to later
\end{equation}

%%%%%%%%%%%%%%%%%%%%%%%%%%%%%%
%Tables: See: https://www.tablesgenerator.com/ and https://www.overleaf.com/learn/latex/Tables
%%%%%%%%%%%%%%%%%%%%%%%%%%%%%%

\begin {table}[h!]
\begin{center}
\label{tab:cells} %<- give the figure a name to refer to later
\caption{Table name}
\begin{tabular}{ |c|c|c| } 
 \hline
 cell1 & cell2 & cell3 \\  \hline
 cell4 & cell5 & cell6 \\  \hline
 cell7 & cell8 & cell9 \\  \hline
\end{tabular}
\end{center}
\end{table}

\subsubsection{Subsubsection Heading Here}
Subsubsection text here. 
\lipsum[10] % <- remove this nonsense text



\section{Conclusion}
The conclusion goes here.
\lipsum[10] % <- remove this nonsense text

\begin{thebibliography}{1}

%Add your references below (I'm not too fussed if this isn't perfect)
\bibitem{examplecite}
H.~Kopka and P.~W. Daly, \emph{A Guide to \LaTeX}, 3rd~ed.\hskip 1em plus
  0.5em minus 0.4em\relax Harlow, England: Addison-Wesley, 1999.

\end{thebibliography}


\end{document}


